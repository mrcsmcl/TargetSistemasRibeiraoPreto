\documentclass{assignment}
\usepackage[T1]{fontenc}            
\usepackage[brazil]{babel}
\usepackage[utf8]{inputenc} 
\usepackage{listings}
\usepackage{amsmath} % pacote para utilizar o ambiente equation*
\usepackage{listings} % for including code
\lstset{
	language=R,   % R code
	literate=
	{á}{{\'a}}1
	{à}{{\`a}}1
	{ã}{{\~a}}1
	{é}{{\'e}}1
	{ê}{{\^e}}1
	{í}{{\'i}}1
	{ó}{{\'o}}1
	{õ}{{\~o}}1
	{ú}{{\'u}}1
	{ü}{{\"u}}1
	{ç}{{\c{c}}}1
}


\begin{document}

\assignmentTitle{Marcos Candido}{}{}{\newline\newline\newline Target Sistemas}{Respostas}

\section*{QUESTÃO 1}

O valor final da variável SOMA será 91.

O trecho de código apresentado utiliza um laço de repetição do tipo "enquanto" para calcular a soma dos números inteiros de 1 a 13.

Inicialmente, as variáveis INDICE, SOMA e K são inicializadas com os valores 13, 0 e 0, respectivamente. Em seguida, o laço de repetição "enquanto" é executado enquanto a variável K for menor que INDICE. Dentro do laço, o valor de K é incrementado em 1 a cada iteração e o valor de SOMA é atualizado somando-se a ele o valor de K.

Dessa forma, o laço executa 13 iterações, somando os números inteiros de 1 a 13. Ao final do laço, o valor final da variável SOMA será a soma de todos esses números, que pode ser calculado usando uma progressão aritmética. Portanto, o valor final da variável SOMA será 91.

\section*{QUESTÃO 2}

O código foi implementado em C\# no Visual Studio na Solução TargetSistemasRibeirãoPreto, que se encontra neste repositório, porém, também, segue o código abaixo:
\\

\lstinputlisting[language=C]{../PERGUNTA2/Program.cs}

\section*{QUESTÃO 3}

a) O próximo número é 9. A sequência é formada por números ímpares, aumentando de 2 em 2.
\newline

$b_n = 2n - 1$

$b_1 = 2\cdot1 - 1 = 1$

$b_2 = 2\cdot2 - 1 = 3$

$b_3 = 2\cdot3 - 1 = 5$

$b_4 = 2\cdot4 - 1 = 7$

$b_5 = 2\cdot5 - 1 = 9$ (próximo número)
\newline

b) O próximo número é 128. A sequência é formada por potências de 2, dobrando a cada termo.
\newline

$c_n = 2^n$

$c_1 = 2^1 = 2$

$c_2 = 2^2 = 4$

$c_3 = 2^3 = 8$

$c_4 = 2^4 = 16$

$c_5 = 2^5 = 32$

$c_6 = 2^6 = 64$

$c_7 = 2^7 = 128$ (próximo número)
\newpage

c) O próximo número é 49. A sequência é formada pelos quadrados dos números naturais, começando em 0.
\newline

$d_n = n^2$

$d_0 = 0^2 = 0$

$d_1 = 1^2 = 1$

$d_2 = 2^2 = 4$

$d_3 = 3^2 = 9$

$d_4 = 4^2 = 16$

$d_5 = 5^2 = 25$

$d_6 = 6^2 = 36$

$d_7 = 7^2 = 49$ (próximo número)
\newline

d) O próximo número é 100. A sequência é formada pelos quadrados dos números pares, começando em 2.
\newline

$e_n = (2n)^2$

$e_1 = (2\cdot1)^2 = 4$

$e_2 = (2\cdot2)^2 = 16$

$e_3 = (2\cdot3)^2 = 36$

$e_4 = (2\cdot4)^2 = 64$

$e_5 = (2\cdot5)^2 = 100$ (próximo número)
\newline

e) O próximo número é 13. A sequência é formada pela soma dos dois números anteriores, começando com 1 e 1 (sequência de Fibonacci).
\newline

$f_n = f_{n-1} + f_{n-2}$

$f_1 = 1$

$f_2 = 1$

$f_3 = 1 + 1 = 2$

$f_4 = 1 + 2 = 3$

$f_5 = 2 + 3 = 5$

$f_6 = 3 + 5 = 8$

$f_7 = 5 + 8 = 13$ (próximo número)
\newline

f) O próximo número é 20. A sequência parece não seguir um padrão claro, mas é possível notar que cada número é o número anterior mais algum valor. Podemos observar que o valor que é adicionado começa em 8 e diminui por 2 a cada termo. Portanto, temos a seguinte sequência: 2, 10, 12, 16, 17, 18, 19, 20.
\newline

\section*{QUESTÃO 4}

Os dois veículos (o carro e o caminhão) estão se aproximando um do outro em direção ao ponto de encontro na estrada. Quando eles se encontram, ambos estarão a mesma distância da cidade de Ribeirão Preto.

No entanto, a distância que cada um dos veículos percorreu para chegar ao ponto de encontro é diferente. O carro estava mais distante de Ribeirão Preto, então percorreu a distância em velocidade maior para chegar ao ponto de encontro. O caminhão, por sua vez, estava mais perto de Ribeirão Preto, então percorreu uma distância menor em uma velocidade menor.

Mas, apesar de terem percorrido distâncias diferentes para chegar ao ponto de encontro, quando eles se cruzarem estarão na mesma distância da cidade de Ribeirão Preto.

\section*{QUESTÃO 5}

O código foi implementado em C\# no Visual Studio na Solução TargetSistemasRibeirãoPreto, que se encontra neste repositório, porém, também, segue o código abaixo:
\\

\lstinputlisting[language=C, inputencoding=utf8]{../PERGUNTA5/Program.cs}

\end{document}
